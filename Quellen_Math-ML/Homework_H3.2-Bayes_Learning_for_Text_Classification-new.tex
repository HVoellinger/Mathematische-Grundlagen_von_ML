\documentclass[11pt]{article}

    \usepackage[breakable]{tcolorbox}
    \usepackage{parskip} % Stop auto-indenting (to mimic markdown behaviour)
    
    \usepackage{iftex}
    \ifPDFTeX
    	\usepackage[T1]{fontenc}
    	\usepackage{mathpazo}
    \else
    	\usepackage{fontspec}
    \fi

    % Basic figure setup, for now with no caption control since it's done
    % automatically by Pandoc (which extracts ![](path) syntax from Markdown).
    \usepackage{graphicx}
    % Maintain compatibility with old templates. Remove in nbconvert 6.0
    \let\Oldincludegraphics\includegraphics
    % Ensure that by default, figures have no caption (until we provide a
    % proper Figure object with a Caption API and a way to capture that
    % in the conversion process - todo).
    \usepackage{caption}
    \DeclareCaptionFormat{nocaption}{}
    \captionsetup{format=nocaption,aboveskip=0pt,belowskip=0pt}

    \usepackage[Export]{adjustbox} % Used to constrain images to a maximum size
    \adjustboxset{max size={0.9\linewidth}{0.9\paperheight}}
    \usepackage{float}
    \floatplacement{figure}{H} % forces figures to be placed at the correct location
    \usepackage{xcolor} % Allow colors to be defined
    \usepackage{enumerate} % Needed for markdown enumerations to work
    \usepackage{geometry} % Used to adjust the document margins
    \usepackage{amsmath} % Equations
    \usepackage{amssymb} % Equations
    \usepackage{textcomp} % defines textquotesingle
    % Hack from http://tex.stackexchange.com/a/47451/13684:
    \AtBeginDocument{%
        \def\PYZsq{\textquotesingle}% Upright quotes in Pygmentized code
    }
    \usepackage{upquote} % Upright quotes for verbatim code
    \usepackage{eurosym} % defines \euro
    \usepackage[mathletters]{ucs} % Extended unicode (utf-8) support
    \usepackage{fancyvrb} % verbatim replacement that allows latex
    \usepackage{grffile} % extends the file name processing of package graphics 
                         % to support a larger range
    \makeatletter % fix for grffile with XeLaTeX
    \def\Gread@@xetex#1{%
      \IfFileExists{"\Gin@base".bb}%
      {\Gread@eps{\Gin@base.bb}}%
      {\Gread@@xetex@aux#1}%
    }
    \makeatother

    % The hyperref package gives us a pdf with properly built
    % internal navigation ('pdf bookmarks' for the table of contents,
    % internal cross-reference links, web links for URLs, etc.)
    \usepackage{hyperref}
    % The default LaTeX title has an obnoxious amount of whitespace. By default,
    % titling removes some of it. It also provides customization options.
    \usepackage{titling}
    \usepackage{longtable} % longtable support required by pandoc >1.10
    \usepackage{booktabs}  % table support for pandoc > 1.12.2
    \usepackage[inline]{enumitem} % IRkernel/repr support (it uses the enumerate* environment)
    \usepackage[normalem]{ulem} % ulem is needed to support strikethroughs (\sout)
                                % normalem makes italics be italics, not underlines
    \usepackage{mathrsfs}
    

    
    % Colors for the hyperref package
    \definecolor{urlcolor}{rgb}{0,.145,.698}
    \definecolor{linkcolor}{rgb}{.71,0.21,0.01}
    \definecolor{citecolor}{rgb}{.12,.54,.11}

    % ANSI colors
    \definecolor{ansi-black}{HTML}{3E424D}
    \definecolor{ansi-black-intense}{HTML}{282C36}
    \definecolor{ansi-red}{HTML}{E75C58}
    \definecolor{ansi-red-intense}{HTML}{B22B31}
    \definecolor{ansi-green}{HTML}{00A250}
    \definecolor{ansi-green-intense}{HTML}{007427}
    \definecolor{ansi-yellow}{HTML}{DDB62B}
    \definecolor{ansi-yellow-intense}{HTML}{B27D12}
    \definecolor{ansi-blue}{HTML}{208FFB}
    \definecolor{ansi-blue-intense}{HTML}{0065CA}
    \definecolor{ansi-magenta}{HTML}{D160C4}
    \definecolor{ansi-magenta-intense}{HTML}{A03196}
    \definecolor{ansi-cyan}{HTML}{60C6C8}
    \definecolor{ansi-cyan-intense}{HTML}{258F8F}
    \definecolor{ansi-white}{HTML}{C5C1B4}
    \definecolor{ansi-white-intense}{HTML}{A1A6B2}
    \definecolor{ansi-default-inverse-fg}{HTML}{FFFFFF}
    \definecolor{ansi-default-inverse-bg}{HTML}{000000}

    % commands and environments needed by pandoc snippets
    % extracted from the output of `pandoc -s`
    \providecommand{\tightlist}{%
      \setlength{\itemsep}{0pt}\setlength{\parskip}{0pt}}
    \DefineVerbatimEnvironment{Highlighting}{Verbatim}{commandchars=\\\{\}}
    % Add ',fontsize=\small' for more characters per line
    \newenvironment{Shaded}{}{}
    \newcommand{\KeywordTok}[1]{\textcolor[rgb]{0.00,0.44,0.13}{\textbf{{#1}}}}
    \newcommand{\DataTypeTok}[1]{\textcolor[rgb]{0.56,0.13,0.00}{{#1}}}
    \newcommand{\DecValTok}[1]{\textcolor[rgb]{0.25,0.63,0.44}{{#1}}}
    \newcommand{\BaseNTok}[1]{\textcolor[rgb]{0.25,0.63,0.44}{{#1}}}
    \newcommand{\FloatTok}[1]{\textcolor[rgb]{0.25,0.63,0.44}{{#1}}}
    \newcommand{\CharTok}[1]{\textcolor[rgb]{0.25,0.44,0.63}{{#1}}}
    \newcommand{\StringTok}[1]{\textcolor[rgb]{0.25,0.44,0.63}{{#1}}}
    \newcommand{\CommentTok}[1]{\textcolor[rgb]{0.38,0.63,0.69}{\textit{{#1}}}}
    \newcommand{\OtherTok}[1]{\textcolor[rgb]{0.00,0.44,0.13}{{#1}}}
    \newcommand{\AlertTok}[1]{\textcolor[rgb]{1.00,0.00,0.00}{\textbf{{#1}}}}
    \newcommand{\FunctionTok}[1]{\textcolor[rgb]{0.02,0.16,0.49}{{#1}}}
    \newcommand{\RegionMarkerTok}[1]{{#1}}
    \newcommand{\ErrorTok}[1]{\textcolor[rgb]{1.00,0.00,0.00}{\textbf{{#1}}}}
    \newcommand{\NormalTok}[1]{{#1}}
    
    % Additional commands for more recent versions of Pandoc
    \newcommand{\ConstantTok}[1]{\textcolor[rgb]{0.53,0.00,0.00}{{#1}}}
    \newcommand{\SpecialCharTok}[1]{\textcolor[rgb]{0.25,0.44,0.63}{{#1}}}
    \newcommand{\VerbatimStringTok}[1]{\textcolor[rgb]{0.25,0.44,0.63}{{#1}}}
    \newcommand{\SpecialStringTok}[1]{\textcolor[rgb]{0.73,0.40,0.53}{{#1}}}
    \newcommand{\ImportTok}[1]{{#1}}
    \newcommand{\DocumentationTok}[1]{\textcolor[rgb]{0.73,0.13,0.13}{\textit{{#1}}}}
    \newcommand{\AnnotationTok}[1]{\textcolor[rgb]{0.38,0.63,0.69}{\textbf{\textit{{#1}}}}}
    \newcommand{\CommentVarTok}[1]{\textcolor[rgb]{0.38,0.63,0.69}{\textbf{\textit{{#1}}}}}
    \newcommand{\VariableTok}[1]{\textcolor[rgb]{0.10,0.09,0.49}{{#1}}}
    \newcommand{\ControlFlowTok}[1]{\textcolor[rgb]{0.00,0.44,0.13}{\textbf{{#1}}}}
    \newcommand{\OperatorTok}[1]{\textcolor[rgb]{0.40,0.40,0.40}{{#1}}}
    \newcommand{\BuiltInTok}[1]{{#1}}
    \newcommand{\ExtensionTok}[1]{{#1}}
    \newcommand{\PreprocessorTok}[1]{\textcolor[rgb]{0.74,0.48,0.00}{{#1}}}
    \newcommand{\AttributeTok}[1]{\textcolor[rgb]{0.49,0.56,0.16}{{#1}}}
    \newcommand{\InformationTok}[1]{\textcolor[rgb]{0.38,0.63,0.69}{\textbf{\textit{{#1}}}}}
    \newcommand{\WarningTok}[1]{\textcolor[rgb]{0.38,0.63,0.69}{\textbf{\textit{{#1}}}}}
    
    
    % Define a nice break command that doesn't care if a line doesn't already
    % exist.
    \def\br{\hspace*{\fill} \\* }
    % Math Jax compatibility definitions
    \def\gt{>}
    \def\lt{<}
    \let\Oldtex\TeX
    \let\Oldlatex\LaTeX
    \renewcommand{\TeX}{\textrm{\Oldtex}}
    \renewcommand{\LaTeX}{\textrm{\Oldlatex}}
    % Document parameters
    % Document title
    \title{Homework\_H3.2-Bayes\_Learning\_for\_Text\_Classification-new}
    
    
    
    
    
% Pygments definitions
\makeatletter
\def\PY@reset{\let\PY@it=\relax \let\PY@bf=\relax%
    \let\PY@ul=\relax \let\PY@tc=\relax%
    \let\PY@bc=\relax \let\PY@ff=\relax}
\def\PY@tok#1{\csname PY@tok@#1\endcsname}
\def\PY@toks#1+{\ifx\relax#1\empty\else%
    \PY@tok{#1}\expandafter\PY@toks\fi}
\def\PY@do#1{\PY@bc{\PY@tc{\PY@ul{%
    \PY@it{\PY@bf{\PY@ff{#1}}}}}}}
\def\PY#1#2{\PY@reset\PY@toks#1+\relax+\PY@do{#2}}

\expandafter\def\csname PY@tok@w\endcsname{\def\PY@tc##1{\textcolor[rgb]{0.73,0.73,0.73}{##1}}}
\expandafter\def\csname PY@tok@c\endcsname{\let\PY@it=\textit\def\PY@tc##1{\textcolor[rgb]{0.25,0.50,0.50}{##1}}}
\expandafter\def\csname PY@tok@cp\endcsname{\def\PY@tc##1{\textcolor[rgb]{0.74,0.48,0.00}{##1}}}
\expandafter\def\csname PY@tok@k\endcsname{\let\PY@bf=\textbf\def\PY@tc##1{\textcolor[rgb]{0.00,0.50,0.00}{##1}}}
\expandafter\def\csname PY@tok@kp\endcsname{\def\PY@tc##1{\textcolor[rgb]{0.00,0.50,0.00}{##1}}}
\expandafter\def\csname PY@tok@kt\endcsname{\def\PY@tc##1{\textcolor[rgb]{0.69,0.00,0.25}{##1}}}
\expandafter\def\csname PY@tok@o\endcsname{\def\PY@tc##1{\textcolor[rgb]{0.40,0.40,0.40}{##1}}}
\expandafter\def\csname PY@tok@ow\endcsname{\let\PY@bf=\textbf\def\PY@tc##1{\textcolor[rgb]{0.67,0.13,1.00}{##1}}}
\expandafter\def\csname PY@tok@nb\endcsname{\def\PY@tc##1{\textcolor[rgb]{0.00,0.50,0.00}{##1}}}
\expandafter\def\csname PY@tok@nf\endcsname{\def\PY@tc##1{\textcolor[rgb]{0.00,0.00,1.00}{##1}}}
\expandafter\def\csname PY@tok@nc\endcsname{\let\PY@bf=\textbf\def\PY@tc##1{\textcolor[rgb]{0.00,0.00,1.00}{##1}}}
\expandafter\def\csname PY@tok@nn\endcsname{\let\PY@bf=\textbf\def\PY@tc##1{\textcolor[rgb]{0.00,0.00,1.00}{##1}}}
\expandafter\def\csname PY@tok@ne\endcsname{\let\PY@bf=\textbf\def\PY@tc##1{\textcolor[rgb]{0.82,0.25,0.23}{##1}}}
\expandafter\def\csname PY@tok@nv\endcsname{\def\PY@tc##1{\textcolor[rgb]{0.10,0.09,0.49}{##1}}}
\expandafter\def\csname PY@tok@no\endcsname{\def\PY@tc##1{\textcolor[rgb]{0.53,0.00,0.00}{##1}}}
\expandafter\def\csname PY@tok@nl\endcsname{\def\PY@tc##1{\textcolor[rgb]{0.63,0.63,0.00}{##1}}}
\expandafter\def\csname PY@tok@ni\endcsname{\let\PY@bf=\textbf\def\PY@tc##1{\textcolor[rgb]{0.60,0.60,0.60}{##1}}}
\expandafter\def\csname PY@tok@na\endcsname{\def\PY@tc##1{\textcolor[rgb]{0.49,0.56,0.16}{##1}}}
\expandafter\def\csname PY@tok@nt\endcsname{\let\PY@bf=\textbf\def\PY@tc##1{\textcolor[rgb]{0.00,0.50,0.00}{##1}}}
\expandafter\def\csname PY@tok@nd\endcsname{\def\PY@tc##1{\textcolor[rgb]{0.67,0.13,1.00}{##1}}}
\expandafter\def\csname PY@tok@s\endcsname{\def\PY@tc##1{\textcolor[rgb]{0.73,0.13,0.13}{##1}}}
\expandafter\def\csname PY@tok@sd\endcsname{\let\PY@it=\textit\def\PY@tc##1{\textcolor[rgb]{0.73,0.13,0.13}{##1}}}
\expandafter\def\csname PY@tok@si\endcsname{\let\PY@bf=\textbf\def\PY@tc##1{\textcolor[rgb]{0.73,0.40,0.53}{##1}}}
\expandafter\def\csname PY@tok@se\endcsname{\let\PY@bf=\textbf\def\PY@tc##1{\textcolor[rgb]{0.73,0.40,0.13}{##1}}}
\expandafter\def\csname PY@tok@sr\endcsname{\def\PY@tc##1{\textcolor[rgb]{0.73,0.40,0.53}{##1}}}
\expandafter\def\csname PY@tok@ss\endcsname{\def\PY@tc##1{\textcolor[rgb]{0.10,0.09,0.49}{##1}}}
\expandafter\def\csname PY@tok@sx\endcsname{\def\PY@tc##1{\textcolor[rgb]{0.00,0.50,0.00}{##1}}}
\expandafter\def\csname PY@tok@m\endcsname{\def\PY@tc##1{\textcolor[rgb]{0.40,0.40,0.40}{##1}}}
\expandafter\def\csname PY@tok@gh\endcsname{\let\PY@bf=\textbf\def\PY@tc##1{\textcolor[rgb]{0.00,0.00,0.50}{##1}}}
\expandafter\def\csname PY@tok@gu\endcsname{\let\PY@bf=\textbf\def\PY@tc##1{\textcolor[rgb]{0.50,0.00,0.50}{##1}}}
\expandafter\def\csname PY@tok@gd\endcsname{\def\PY@tc##1{\textcolor[rgb]{0.63,0.00,0.00}{##1}}}
\expandafter\def\csname PY@tok@gi\endcsname{\def\PY@tc##1{\textcolor[rgb]{0.00,0.63,0.00}{##1}}}
\expandafter\def\csname PY@tok@gr\endcsname{\def\PY@tc##1{\textcolor[rgb]{1.00,0.00,0.00}{##1}}}
\expandafter\def\csname PY@tok@ge\endcsname{\let\PY@it=\textit}
\expandafter\def\csname PY@tok@gs\endcsname{\let\PY@bf=\textbf}
\expandafter\def\csname PY@tok@gp\endcsname{\let\PY@bf=\textbf\def\PY@tc##1{\textcolor[rgb]{0.00,0.00,0.50}{##1}}}
\expandafter\def\csname PY@tok@go\endcsname{\def\PY@tc##1{\textcolor[rgb]{0.53,0.53,0.53}{##1}}}
\expandafter\def\csname PY@tok@gt\endcsname{\def\PY@tc##1{\textcolor[rgb]{0.00,0.27,0.87}{##1}}}
\expandafter\def\csname PY@tok@err\endcsname{\def\PY@bc##1{\setlength{\fboxsep}{0pt}\fcolorbox[rgb]{1.00,0.00,0.00}{1,1,1}{\strut ##1}}}
\expandafter\def\csname PY@tok@kc\endcsname{\let\PY@bf=\textbf\def\PY@tc##1{\textcolor[rgb]{0.00,0.50,0.00}{##1}}}
\expandafter\def\csname PY@tok@kd\endcsname{\let\PY@bf=\textbf\def\PY@tc##1{\textcolor[rgb]{0.00,0.50,0.00}{##1}}}
\expandafter\def\csname PY@tok@kn\endcsname{\let\PY@bf=\textbf\def\PY@tc##1{\textcolor[rgb]{0.00,0.50,0.00}{##1}}}
\expandafter\def\csname PY@tok@kr\endcsname{\let\PY@bf=\textbf\def\PY@tc##1{\textcolor[rgb]{0.00,0.50,0.00}{##1}}}
\expandafter\def\csname PY@tok@bp\endcsname{\def\PY@tc##1{\textcolor[rgb]{0.00,0.50,0.00}{##1}}}
\expandafter\def\csname PY@tok@fm\endcsname{\def\PY@tc##1{\textcolor[rgb]{0.00,0.00,1.00}{##1}}}
\expandafter\def\csname PY@tok@vc\endcsname{\def\PY@tc##1{\textcolor[rgb]{0.10,0.09,0.49}{##1}}}
\expandafter\def\csname PY@tok@vg\endcsname{\def\PY@tc##1{\textcolor[rgb]{0.10,0.09,0.49}{##1}}}
\expandafter\def\csname PY@tok@vi\endcsname{\def\PY@tc##1{\textcolor[rgb]{0.10,0.09,0.49}{##1}}}
\expandafter\def\csname PY@tok@vm\endcsname{\def\PY@tc##1{\textcolor[rgb]{0.10,0.09,0.49}{##1}}}
\expandafter\def\csname PY@tok@sa\endcsname{\def\PY@tc##1{\textcolor[rgb]{0.73,0.13,0.13}{##1}}}
\expandafter\def\csname PY@tok@sb\endcsname{\def\PY@tc##1{\textcolor[rgb]{0.73,0.13,0.13}{##1}}}
\expandafter\def\csname PY@tok@sc\endcsname{\def\PY@tc##1{\textcolor[rgb]{0.73,0.13,0.13}{##1}}}
\expandafter\def\csname PY@tok@dl\endcsname{\def\PY@tc##1{\textcolor[rgb]{0.73,0.13,0.13}{##1}}}
\expandafter\def\csname PY@tok@s2\endcsname{\def\PY@tc##1{\textcolor[rgb]{0.73,0.13,0.13}{##1}}}
\expandafter\def\csname PY@tok@sh\endcsname{\def\PY@tc##1{\textcolor[rgb]{0.73,0.13,0.13}{##1}}}
\expandafter\def\csname PY@tok@s1\endcsname{\def\PY@tc##1{\textcolor[rgb]{0.73,0.13,0.13}{##1}}}
\expandafter\def\csname PY@tok@mb\endcsname{\def\PY@tc##1{\textcolor[rgb]{0.40,0.40,0.40}{##1}}}
\expandafter\def\csname PY@tok@mf\endcsname{\def\PY@tc##1{\textcolor[rgb]{0.40,0.40,0.40}{##1}}}
\expandafter\def\csname PY@tok@mh\endcsname{\def\PY@tc##1{\textcolor[rgb]{0.40,0.40,0.40}{##1}}}
\expandafter\def\csname PY@tok@mi\endcsname{\def\PY@tc##1{\textcolor[rgb]{0.40,0.40,0.40}{##1}}}
\expandafter\def\csname PY@tok@il\endcsname{\def\PY@tc##1{\textcolor[rgb]{0.40,0.40,0.40}{##1}}}
\expandafter\def\csname PY@tok@mo\endcsname{\def\PY@tc##1{\textcolor[rgb]{0.40,0.40,0.40}{##1}}}
\expandafter\def\csname PY@tok@ch\endcsname{\let\PY@it=\textit\def\PY@tc##1{\textcolor[rgb]{0.25,0.50,0.50}{##1}}}
\expandafter\def\csname PY@tok@cm\endcsname{\let\PY@it=\textit\def\PY@tc##1{\textcolor[rgb]{0.25,0.50,0.50}{##1}}}
\expandafter\def\csname PY@tok@cpf\endcsname{\let\PY@it=\textit\def\PY@tc##1{\textcolor[rgb]{0.25,0.50,0.50}{##1}}}
\expandafter\def\csname PY@tok@c1\endcsname{\let\PY@it=\textit\def\PY@tc##1{\textcolor[rgb]{0.25,0.50,0.50}{##1}}}
\expandafter\def\csname PY@tok@cs\endcsname{\let\PY@it=\textit\def\PY@tc##1{\textcolor[rgb]{0.25,0.50,0.50}{##1}}}

\def\PYZbs{\char`\\}
\def\PYZus{\char`\_}
\def\PYZob{\char`\{}
\def\PYZcb{\char`\}}
\def\PYZca{\char`\^}
\def\PYZam{\char`\&}
\def\PYZlt{\char`\<}
\def\PYZgt{\char`\>}
\def\PYZsh{\char`\#}
\def\PYZpc{\char`\%}
\def\PYZdl{\char`\$}
\def\PYZhy{\char`\-}
\def\PYZsq{\char`\'}
\def\PYZdq{\char`\"}
\def\PYZti{\char`\~}
% for compatibility with earlier versions
\def\PYZat{@}
\def\PYZlb{[}
\def\PYZrb{]}
\makeatother


    % For linebreaks inside Verbatim environment from package fancyvrb. 
    \makeatletter
        \newbox\Wrappedcontinuationbox 
        \newbox\Wrappedvisiblespacebox 
        \newcommand*\Wrappedvisiblespace {\textcolor{red}{\textvisiblespace}} 
        \newcommand*\Wrappedcontinuationsymbol {\textcolor{red}{\llap{\tiny$\m@th\hookrightarrow$}}} 
        \newcommand*\Wrappedcontinuationindent {3ex } 
        \newcommand*\Wrappedafterbreak {\kern\Wrappedcontinuationindent\copy\Wrappedcontinuationbox} 
        % Take advantage of the already applied Pygments mark-up to insert 
        % potential linebreaks for TeX processing. 
        %        {, <, #, %, $, ' and ": go to next line. 
        %        _, }, ^, &, >, - and ~: stay at end of broken line. 
        % Use of \textquotesingle for straight quote. 
        \newcommand*\Wrappedbreaksatspecials {% 
            \def\PYGZus{\discretionary{\char`\_}{\Wrappedafterbreak}{\char`\_}}% 
            \def\PYGZob{\discretionary{}{\Wrappedafterbreak\char`\{}{\char`\{}}% 
            \def\PYGZcb{\discretionary{\char`\}}{\Wrappedafterbreak}{\char`\}}}% 
            \def\PYGZca{\discretionary{\char`\^}{\Wrappedafterbreak}{\char`\^}}% 
            \def\PYGZam{\discretionary{\char`\&}{\Wrappedafterbreak}{\char`\&}}% 
            \def\PYGZlt{\discretionary{}{\Wrappedafterbreak\char`\<}{\char`\<}}% 
            \def\PYGZgt{\discretionary{\char`\>}{\Wrappedafterbreak}{\char`\>}}% 
            \def\PYGZsh{\discretionary{}{\Wrappedafterbreak\char`\#}{\char`\#}}% 
            \def\PYGZpc{\discretionary{}{\Wrappedafterbreak\char`\%}{\char`\%}}% 
            \def\PYGZdl{\discretionary{}{\Wrappedafterbreak\char`\$}{\char`\$}}% 
            \def\PYGZhy{\discretionary{\char`\-}{\Wrappedafterbreak}{\char`\-}}% 
            \def\PYGZsq{\discretionary{}{\Wrappedafterbreak\textquotesingle}{\textquotesingle}}% 
            \def\PYGZdq{\discretionary{}{\Wrappedafterbreak\char`\"}{\char`\"}}% 
            \def\PYGZti{\discretionary{\char`\~}{\Wrappedafterbreak}{\char`\~}}% 
        } 
        % Some characters . , ; ? ! / are not pygmentized. 
        % This macro makes them "active" and they will insert potential linebreaks 
        \newcommand*\Wrappedbreaksatpunct {% 
            \lccode`\~`\.\lowercase{\def~}{\discretionary{\hbox{\char`\.}}{\Wrappedafterbreak}{\hbox{\char`\.}}}% 
            \lccode`\~`\,\lowercase{\def~}{\discretionary{\hbox{\char`\,}}{\Wrappedafterbreak}{\hbox{\char`\,}}}% 
            \lccode`\~`\;\lowercase{\def~}{\discretionary{\hbox{\char`\;}}{\Wrappedafterbreak}{\hbox{\char`\;}}}% 
            \lccode`\~`\:\lowercase{\def~}{\discretionary{\hbox{\char`\:}}{\Wrappedafterbreak}{\hbox{\char`\:}}}% 
            \lccode`\~`\?\lowercase{\def~}{\discretionary{\hbox{\char`\?}}{\Wrappedafterbreak}{\hbox{\char`\?}}}% 
            \lccode`\~`\!\lowercase{\def~}{\discretionary{\hbox{\char`\!}}{\Wrappedafterbreak}{\hbox{\char`\!}}}% 
            \lccode`\~`\/\lowercase{\def~}{\discretionary{\hbox{\char`\/}}{\Wrappedafterbreak}{\hbox{\char`\/}}}% 
            \catcode`\.\active
            \catcode`\,\active 
            \catcode`\;\active
            \catcode`\:\active
            \catcode`\?\active
            \catcode`\!\active
            \catcode`\/\active 
            \lccode`\~`\~ 	
        }
    \makeatother

    \let\OriginalVerbatim=\Verbatim
    \makeatletter
    \renewcommand{\Verbatim}[1][1]{%
        %\parskip\z@skip
        \sbox\Wrappedcontinuationbox {\Wrappedcontinuationsymbol}%
        \sbox\Wrappedvisiblespacebox {\FV@SetupFont\Wrappedvisiblespace}%
        \def\FancyVerbFormatLine ##1{\hsize\linewidth
            \vtop{\raggedright\hyphenpenalty\z@\exhyphenpenalty\z@
                \doublehyphendemerits\z@\finalhyphendemerits\z@
                \strut ##1\strut}%
        }%
        % If the linebreak is at a space, the latter will be displayed as visible
        % space at end of first line, and a continuation symbol starts next line.
        % Stretch/shrink are however usually zero for typewriter font.
        \def\FV@Space {%
            \nobreak\hskip\z@ plus\fontdimen3\font minus\fontdimen4\font
            \discretionary{\copy\Wrappedvisiblespacebox}{\Wrappedafterbreak}
            {\kern\fontdimen2\font}%
        }%
        
        % Allow breaks at special characters using \PYG... macros.
        \Wrappedbreaksatspecials
        % Breaks at punctuation characters . , ; ? ! and / need catcode=\active 	
        \OriginalVerbatim[#1,codes*=\Wrappedbreaksatpunct]%
    }
    \makeatother

    % Exact colors from NB
    \definecolor{incolor}{HTML}{303F9F}
    \definecolor{outcolor}{HTML}{D84315}
    \definecolor{cellborder}{HTML}{CFCFCF}
    \definecolor{cellbackground}{HTML}{F7F7F7}
    
    % prompt
    \makeatletter
    \newcommand{\boxspacing}{\kern\kvtcb@left@rule\kern\kvtcb@boxsep}
    \makeatother
    \newcommand{\prompt}[4]{
        \ttfamily\llap{{\color{#2}[#3]:\hspace{3pt}#4}}\vspace{-\baselineskip}
    }
    

    
    % Prevent overflowing lines due to hard-to-break entities
    \sloppy 
    % Setup hyperref package
    \hypersetup{
      breaklinks=true,  % so long urls are correctly broken across lines
      colorlinks=true,
      urlcolor=urlcolor,
      linkcolor=linkcolor,
      citecolor=citecolor,
      }
    % Slightly bigger margins than the latex defaults
    
    \geometry{verbose,tmargin=1in,bmargin=1in,lmargin=1in,rmargin=1in}
    
    

\begin{document}
    
    \maketitle
    
    

    
    \hypertarget{naive-bayes-text-classification}{%
\section{Naive Bayes Text
Classification}\label{naive-bayes-text-classification}}

\hypertarget{sentence-classification-using-naive-bayes-algorithm}{%
\subsubsection{Sentence Classification using Naive Bayes
Algorithm}\label{sentence-classification-using-naive-bayes-algorithm}}

We made a simple Algorithm to try and classify senteces into either
Sports or Not Sports sentences. We start with a couple sentences either
classed ``Sports'' or ``Not Sports'' and try to classify new sentences
based on that. At the end we make a comparison, which class (``Sports''
or ``Not Sports'') the new sentence is more likely to end up in.

\hypertarget{copyright}{%
\subsection{Copyright}\label{copyright}}

This Jupyter Notebook was primarily created as solution to an exercise
in the lectute ``Introduction to Machine Learning'' (Dr.~Hermann
Völlinger), DHBW Stuttgart, WS 2020 The first version was created by the
two students Alireza Gholami and Jannik Schwarz in October 2020 Later
versions are extended and completetd by Dr.~Hermann Völlinger Actual
version see saving date of the notebook

\hypertarget{machine-learnig-ml-model-method}{%
\subsection{Machine Learnig (ML) Model /
Method}\label{machine-learnig-ml-model-method}}

Important for a ML solution is the algorithm which is used for our
solution. In this example we use the algorithm we learned in the
lecture: ``Sentence Classification'' using ``Naive Bayes Algorithm''

For more information see the slides:
``Homework\_H3.2-Bayes\_Learning\_for\_Text\_Classification-Folien.pdf''

\hypertarget{what-happens-here}{%
\subsection{What happens here:}\label{what-happens-here}}

\begin{verbatim}
1. Import the Sklearn libraries which we need
2. Provide training data and do transformations.
3. Create dictionaries and count the words in each class.
4. Calculate probabilities of the words.
\end{verbatim}

To evaluate a new sentence\ldots{}

\begin{verbatim}
5. Vectorize and transform all sentences
6. Count all words
7. Transform new sentence
8. Perform Laplace Smoothing, so we don't multiply with 0
9. Calculate probability of the new sentence for each class
10. Output whats more likely
\end{verbatim}

    \begin{tcolorbox}[breakable, size=fbox, boxrule=1pt, pad at break*=1mm,colback=cellbackground, colframe=cellborder]
\prompt{In}{incolor}{3}{\boxspacing}
\begin{Verbatim}[commandchars=\\\{\}]
\PY{c+c1}{\PYZsh{} This notebook was created by Alireza Gholami and Jannik Schwarz}

\PY{n+nb}{print}\PY{p}{(}\PY{l+s+s1}{\PYZsq{}}\PY{l+s+s1}{***********************************************************************}\PY{l+s+s1}{\PYZsq{}}\PY{p}{)}
\PY{n+nb}{print}\PY{p}{(}\PY{l+s+s1}{\PYZsq{}}\PY{l+s+s1}{This Jupyter Notebook was primarily created as solution to an exercise }\PY{l+s+s1}{\PYZsq{}}\PY{p}{)} 
\PY{n+nb}{print}\PY{p}{(}\PY{l+s+s1}{\PYZsq{}}\PY{l+s+s1}{in the lecture: }\PY{l+s+s1}{\PYZdq{}}\PY{l+s+s1}{Introduction to Machine Learning}\PY{l+s+s1}{\PYZdq{}}\PY{l+s+s1}{(Dr. Hermann Völlinger)}\PY{l+s+s1}{\PYZsq{}}\PY{p}{)}
\PY{n+nb}{print}\PY{p}{(}\PY{l+s+s1}{\PYZsq{}}\PY{l+s+s1}{DHBW Stuttgart, WS 2020. First version was created by Alireza Gholami }\PY{l+s+s1}{\PYZsq{}}\PY{p}{)}
\PY{n+nb}{print}\PY{p}{(}\PY{l+s+s1}{\PYZsq{}}\PY{l+s+s1}{and Jannik Schwarz in October 2020. Later versions are extended by Dr.}\PY{l+s+s1}{\PYZsq{}}\PY{p}{)}
\PY{n+nb}{print}\PY{p}{(}\PY{l+s+s1}{\PYZsq{}}\PY{l+s+s1}{Hermann Völlinger, see actual date of notebook }\PY{l+s+s1}{\PYZsq{}}\PY{p}{)}
\PY{n+nb}{print}\PY{p}{(}\PY{l+s+s1}{\PYZsq{}}\PY{l+s+s1}{*************************************************************************}\PY{l+s+s1}{\PYZsq{}}\PY{p}{)}
\PY{n+nb}{print}\PY{p}{(}\PY{l+s+s1}{\PYZsq{}}\PY{l+s+s1}{Method: }\PY{l+s+s1}{\PYZdq{}}\PY{l+s+s1}{Sentence classification}\PY{l+s+s1}{\PYZdq{}}\PY{l+s+s1}{ using }\PY{l+s+s1}{\PYZdq{}}\PY{l+s+s1}{Naive Bayes Algorithm}\PY{l+s+s1}{\PYZdq{}}\PY{l+s+s1}{, see the }\PY{l+s+s1}{\PYZsq{}}\PY{p}{)} 
\PY{n+nb}{print}\PY{p}{(}\PY{l+s+s1}{\PYZsq{}}\PY{l+s+s1}{slides: }\PY{l+s+s1}{\PYZdq{}}\PY{l+s+s1}{Homework\PYZus{}H3.2\PYZhy{}Bayes\PYZus{}Learning\PYZus{}for\PYZus{}Text\PYZus{}Classification\PYZhy{}Folien.pdf}\PY{l+s+s1}{\PYZdq{}}\PY{l+s+s1}{\PYZsq{}}\PY{p}{)}
\PY{n+nb}{print}\PY{p}{(}\PY{l+s+s1}{\PYZsq{}}\PY{l+s+s1}{*************************************************************************}\PY{l+s+s1}{\PYZsq{}}\PY{p}{)}

\PY{c+c1}{\PYZsh{} Importing everything we need}
\PY{k+kn}{import} \PY{n+nn}{pandas} \PY{k}{as} \PY{n+nn}{pd}
\PY{k+kn}{from} \PY{n+nn}{sklearn}\PY{n+nn}{.}\PY{n+nn}{feature\PYZus{}extraction}\PY{n+nn}{.}\PY{n+nn}{text} \PY{k+kn}{import} \PY{n}{CountVectorizer}
\PY{k+kn}{from} \PY{n+nn}{nltk}\PY{n+nn}{.}\PY{n+nn}{tokenize} \PY{k+kn}{import} \PY{n}{word\PYZus{}tokenize}

\PY{c+c1}{\PYZsh{} Import libary time to check execution date+time}
\PY{k+kn}{import} \PY{n+nn}{time}
\PY{c+c1}{\PYZsh{} print the date \PYZam{} time of the notebook}
\PY{n+nb}{print}\PY{p}{(}\PY{l+s+s1}{\PYZsq{}}\PY{l+s+s1}{************************************************************************}\PY{l+s+s1}{\PYZsq{}}\PY{p}{)}
\PY{n+nb}{print}\PY{p}{(}\PY{l+s+s2}{\PYZdq{}}\PY{l+s+s2}{Actual date \PYZam{} time of the notebook:}\PY{l+s+s2}{\PYZdq{}}\PY{p}{,}\PY{n}{time}\PY{o}{.}\PY{n}{strftime}\PY{p}{(}\PY{l+s+s2}{\PYZdq{}}\PY{l+s+si}{\PYZpc{}d}\PY{l+s+s2}{.}\PY{l+s+s2}{\PYZpc{}}\PY{l+s+s2}{m.}\PY{l+s+s2}{\PYZpc{}}\PY{l+s+s2}{Y }\PY{l+s+s2}{\PYZpc{}}\PY{l+s+s2}{H:}\PY{l+s+s2}{\PYZpc{}}\PY{l+s+s2}{M:}\PY{l+s+s2}{\PYZpc{}}\PY{l+s+s2}{S}\PY{l+s+s2}{\PYZdq{}}\PY{p}{)}\PY{p}{)}
\PY{n+nb}{print}\PY{p}{(}\PY{l+s+s1}{\PYZsq{}}\PY{l+s+s1}{************************************************************************}\PY{l+s+s1}{\PYZsq{}}\PY{p}{)}

\PY{c+c1}{\PYZsh{}check versions of libraries}
\PY{n+nb}{print}\PY{p}{(}\PY{l+s+s1}{\PYZsq{}}\PY{l+s+s1}{pandas version is: }\PY{l+s+si}{\PYZob{}\PYZcb{}}\PY{l+s+s1}{\PYZsq{}}\PY{o}{.}\PY{n}{format}\PY{p}{(}\PY{n}{pd}\PY{o}{.}\PY{n}{\PYZus{}\PYZus{}version\PYZus{}\PYZus{}}\PY{p}{)}\PY{p}{)}    

\PY{k+kn}{import} \PY{n+nn}{sklearn} 
\PY{k+kn}{import} \PY{n+nn}{nltk}
\PY{n+nb}{print}\PY{p}{(}\PY{l+s+s1}{\PYZsq{}}\PY{l+s+s1}{sklearn version is: }\PY{l+s+si}{\PYZob{}\PYZcb{}}\PY{l+s+s1}{\PYZsq{}}\PY{o}{.}\PY{n}{format}\PY{p}{(}\PY{n}{sklearn}\PY{o}{.}\PY{n}{\PYZus{}\PYZus{}version\PYZus{}\PYZus{}}\PY{p}{)}\PY{p}{)} 
\PY{n+nb}{print}\PY{p}{(}\PY{l+s+s1}{\PYZsq{}}\PY{l+s+s1}{nltk version is: }\PY{l+s+si}{\PYZob{}\PYZcb{}}\PY{l+s+s1}{\PYZsq{}}\PY{o}{.}\PY{n}{format}\PY{p}{(}\PY{n}{nltk}\PY{o}{.}\PY{n}{\PYZus{}\PYZus{}version\PYZus{}\PYZus{}}\PY{p}{)}\PY{p}{)} 
\end{Verbatim}
\end{tcolorbox}

    \begin{Verbatim}[commandchars=\\\{\}]
***********************************************************************
This Jupyter Notebook was primarily created as solution to an exercise
in the lecture: "Introduction to Machine Learning"(Dr. Hermann Völlinger)
DHBW Stuttgart, WS 2020. First version was created by Alireza Gholami
and Jannik Schwarz in October 2020. Later versions are extended by Dr.
Hermann Völlinger, see actual date of notebook
*************************************************************************
Method: "Sentence classification" using "Naive Bayes Algorithm", see the
slides: "Homework\_H3.2-Bayes\_Learning\_for\_Text\_Classification-Folien.pdf"
*************************************************************************
************************************************************************
Actual date \& time of the notebook: 24.08.2023 21:01:09
************************************************************************
pandas version is: 1.0.1
sklearn version is: 0.22.1
nltk version is: 3.4.5
    \end{Verbatim}

    \begin{tcolorbox}[breakable, size=fbox, boxrule=1pt, pad at break*=1mm,colback=cellbackground, colframe=cellborder]
\prompt{In}{incolor}{4}{\boxspacing}
\begin{Verbatim}[commandchars=\\\{\}]
\PY{c+c1}{\PYZsh{} Naming the two columns of the matrix}
\PY{n}{columns} \PY{o}{=} \PY{p}{[}\PY{l+s+s1}{\PYZsq{}}\PY{l+s+s1}{sentence}\PY{l+s+s1}{\PYZsq{}}\PY{p}{,} \PY{l+s+s1}{\PYZsq{}}\PY{l+s+s1}{class}\PY{l+s+s1}{\PYZsq{}}\PY{p}{]}

\PY{c+c1}{\PYZsh{} Our training data consists of six labeled sentences}
\PY{n}{rows} \PY{o}{=} \PY{p}{[}\PY{p}{[}\PY{l+s+s1}{\PYZsq{}}\PY{l+s+s1}{A great game}\PY{l+s+s1}{\PYZsq{}}\PY{p}{,} \PY{l+s+s1}{\PYZsq{}}\PY{l+s+s1}{Sports}\PY{l+s+s1}{\PYZsq{}}\PY{p}{]}\PY{p}{,}
        \PY{p}{[}\PY{l+s+s1}{\PYZsq{}}\PY{l+s+s1}{The election was over}\PY{l+s+s1}{\PYZsq{}}\PY{p}{,} \PY{l+s+s1}{\PYZsq{}}\PY{l+s+s1}{Not Sports}\PY{l+s+s1}{\PYZsq{}}\PY{p}{]}\PY{p}{,}
        \PY{p}{[}\PY{l+s+s1}{\PYZsq{}}\PY{l+s+s1}{Very clean match}\PY{l+s+s1}{\PYZsq{}}\PY{p}{,} \PY{l+s+s1}{\PYZsq{}}\PY{l+s+s1}{Sports}\PY{l+s+s1}{\PYZsq{}}\PY{p}{]}\PY{p}{,}
        \PY{p}{[}\PY{l+s+s1}{\PYZsq{}}\PY{l+s+s1}{A clean but forgettable game}\PY{l+s+s1}{\PYZsq{}}\PY{p}{,} \PY{l+s+s1}{\PYZsq{}}\PY{l+s+s1}{Sports}\PY{l+s+s1}{\PYZsq{}}\PY{p}{]}\PY{p}{,}
        \PY{p}{[}\PY{l+s+s1}{\PYZsq{}}\PY{l+s+s1}{A very close game}\PY{l+s+s1}{\PYZsq{}}\PY{p}{,} \PY{l+s+s1}{\PYZsq{}}\PY{l+s+s1}{Sports}\PY{l+s+s1}{\PYZsq{}}\PY{p}{]}\PY{p}{]}

\PY{c+c1}{\PYZsh{} we define a dataframe structure for the training data}
\PY{c+c1}{\PYZsh{} we use the Dataframe structure of the pandas library}
\PY{n}{training\PYZus{}data} \PY{o}{=} \PY{n}{pd}\PY{o}{.}\PY{n}{DataFrame}\PY{p}{(}\PY{n}{rows}\PY{p}{,} \PY{n}{columns}\PY{o}{=}\PY{n}{columns}\PY{p}{)}
\PY{n+nb}{print}\PY{p}{(}\PY{l+s+sa}{f}\PY{l+s+s1}{\PYZsq{}}\PY{l+s+s1}{The training data consists of the six labeled sentences:}\PY{l+s+se}{\PYZbs{}n}\PY{l+s+si}{\PYZob{}training\PYZus{}data\PYZcb{}}\PY{l+s+se}{\PYZbs{}n}\PY{l+s+s1}{\PYZsq{}}\PY{p}{)}
\end{Verbatim}
\end{tcolorbox}

    \begin{Verbatim}[commandchars=\\\{\}]
The training data consists of the six labeled sentences:
                       sentence       class
0                  A great game      Sports
1         The election was over  Not Sports
2              Very clean match      Sports
3  A clean but forgettable game      Sports
4             A very close game      Sports

    \end{Verbatim}

    \begin{tcolorbox}[breakable, size=fbox, boxrule=1pt, pad at break*=1mm,colback=cellbackground, colframe=cellborder]
\prompt{In}{incolor}{5}{\boxspacing}
\begin{Verbatim}[commandchars=\\\{\}]
\PY{c+c1}{\PYZsh{} Turns the training data senteneces into vectors }

\PY{k}{def} \PY{n+nf}{vectorisation}\PY{p}{(}\PY{n}{my\PYZus{}class}\PY{p}{)}\PY{p}{:}
    
\PY{c+c1}{\PYZsh{} my\PYZus{}docs contains the sentences for a class (sports or not sports)}
    \PY{n}{my\PYZus{}docs} \PY{o}{=} \PY{p}{[}\PY{n}{row}\PY{p}{[}\PY{l+s+s1}{\PYZsq{}}\PY{l+s+s1}{sentence}\PY{l+s+s1}{\PYZsq{}}\PY{p}{]} \PY{k}{for} \PY{n}{index}\PY{p}{,} \PY{n}{row} \PY{o+ow}{in} \PY{n}{training\PYZus{}data}\PY{o}{.}\PY{n}{iterrows}\PY{p}{(}\PY{p}{)} \PY{k}{if} \PY{n}{row}\PY{p}{[}\PY{l+s+s1}{\PYZsq{}}\PY{l+s+s1}{class}\PY{l+s+s1}{\PYZsq{}}\PY{p}{]} \PY{o}{==} \PY{n}{my\PYZus{}class}\PY{p}{]}
\PY{c+c1}{\PYZsh{} CountVectorizer count the words in each vector, stopword like \PYZdq{}the\PYZdq{} are omitted }
\PY{c+c1}{\PYZsh{} creates a vector that counts the occurence of words in a sentence}
    \PY{n}{my\PYZus{}vector} \PY{o}{=} \PY{n}{CountVectorizer}\PY{p}{(}\PY{n}{token\PYZus{}pattern}\PY{o}{=}\PY{l+s+sa}{r}\PY{l+s+s2}{\PYZdq{}}\PY{l+s+s2}{(?u)}\PY{l+s+s2}{\PYZbs{}}\PY{l+s+s2}{b}\PY{l+s+s2}{\PYZbs{}}\PY{l+s+s2}{w+}\PY{l+s+s2}{\PYZbs{}}\PY{l+s+s2}{b}\PY{l+s+s2}{\PYZdq{}}\PY{p}{)} \PY{c+c1}{\PYZsh{} Token\PYZhy{}Pattern damit einstellige Wörter wie \PYZsq{}a\PYZsq{} gelesen werden}
    
    \PY{c+c1}{\PYZsh{} transform the sentences}
    \PY{n}{my\PYZus{}x} \PY{o}{=} \PY{n}{my\PYZus{}vector}\PY{o}{.}\PY{n}{fit\PYZus{}transform}\PY{p}{(}\PY{n}{my\PYZus{}docs}\PY{p}{)}
    
    \PY{c+c1}{\PYZsh{} tdm = term\PYZus{}document\PYZus{}matrix\PYZus{}sport | create the matrix with the vectors for a class}
    \PY{n}{tdm} \PY{o}{=} \PY{n}{pd}\PY{o}{.}\PY{n}{DataFrame}\PY{p}{(}\PY{n}{my\PYZus{}x}\PY{o}{.}\PY{n}{toarray}\PY{p}{(}\PY{p}{)}\PY{p}{,} \PY{n}{columns}\PY{o}{=}\PY{n}{my\PYZus{}vector}\PY{o}{.}\PY{n}{get\PYZus{}feature\PYZus{}names}\PY{p}{(}\PY{p}{)}\PY{p}{)}
    \PY{k}{return} \PY{n}{tdm}\PY{p}{,} \PY{n}{my\PYZus{}vector}\PY{p}{,} \PY{n}{my\PYZus{}x}
\end{Verbatim}
\end{tcolorbox}

    \begin{tcolorbox}[breakable, size=fbox, boxrule=1pt, pad at break*=1mm,colback=cellbackground, colframe=cellborder]
\prompt{In}{incolor}{6}{\boxspacing}
\begin{Verbatim}[commandchars=\\\{\}]
\PY{c+c1}{\PYZsh{} Here we are actually creating the matrix for sport and not sport sentences}
\PY{n}{tdm\PYZus{}sport}\PY{p}{,} \PY{n}{vector\PYZus{}sport}\PY{p}{,} \PY{n}{X\PYZus{}sport} \PY{o}{=} \PY{n}{vectorisation}\PY{p}{(}\PY{l+s+s1}{\PYZsq{}}\PY{l+s+s1}{Sports}\PY{l+s+s1}{\PYZsq{}}\PY{p}{)}
\PY{n}{tdm\PYZus{}not\PYZus{}sport}\PY{p}{,} \PY{n}{vector\PYZus{}not\PYZus{}sport}\PY{p}{,} \PY{n}{X\PYZus{}not\PYZus{}sport} \PY{o}{=} \PY{n}{vectorisation}\PY{p}{(}\PY{l+s+s1}{\PYZsq{}}\PY{l+s+s1}{Not Sports}\PY{l+s+s1}{\PYZsq{}}\PY{p}{)}

\PY{n+nb}{print}\PY{p}{(}\PY{l+s+sa}{f}\PY{l+s+s1}{\PYZsq{}}\PY{l+s+s1}{Sport sentence matrix: }\PY{l+s+se}{\PYZbs{}n}\PY{l+s+si}{\PYZob{}tdm\PYZus{}sport\PYZcb{}}\PY{l+s+se}{\PYZbs{}n}\PY{l+s+s1}{\PYZsq{}}\PY{p}{)}
\PY{n+nb}{print}\PY{p}{(}\PY{l+s+sa}{f}\PY{l+s+s1}{\PYZsq{}}\PY{l+s+s1}{Not sport sentence matrix: }\PY{l+s+se}{\PYZbs{}n}\PY{l+s+si}{\PYZob{}tdm\PYZus{}not\PYZus{}sport\PYZcb{}}\PY{l+s+se}{\PYZbs{}n}\PY{l+s+s1}{\PYZsq{}}\PY{p}{)}
\PY{n+nb}{print}\PY{p}{(}\PY{l+s+sa}{f}\PY{l+s+s1}{\PYZsq{}}\PY{l+s+s1}{Amount of sport sentences: }\PY{l+s+s1}{\PYZob{}}\PY{l+s+s1}{len(tdm\PYZus{}sport)\PYZcb{}}\PY{l+s+s1}{\PYZsq{}}\PY{p}{)}
\PY{n+nb}{print}\PY{p}{(}\PY{l+s+sa}{f}\PY{l+s+s1}{\PYZsq{}}\PY{l+s+s1}{Amount of not sport senteces: }\PY{l+s+s1}{\PYZob{}}\PY{l+s+s1}{len(tdm\PYZus{}not\PYZus{}sport)\PYZcb{}}\PY{l+s+s1}{\PYZsq{}}\PY{p}{)}
\PY{n+nb}{print}\PY{p}{(}\PY{l+s+sa}{f}\PY{l+s+s1}{\PYZsq{}}\PY{l+s+s1}{Total amount of sentences: }\PY{l+s+s1}{\PYZob{}}\PY{l+s+s1}{len(rows)\PYZcb{}}\PY{l+s+s1}{\PYZsq{}}\PY{p}{)}
\end{Verbatim}
\end{tcolorbox}

    \begin{Verbatim}[commandchars=\\\{\}]
Sport sentence matrix:
   a  but  clean  close  forgettable  game  great  match  very
0  1    0      0      0            0     1      1      0     0
1  0    0      1      0            0     0      0      1     1
2  1    1      1      0            1     1      0      0     0
3  1    0      0      1            0     1      0      0     1

Not sport sentence matrix:
   election  over  the  was
0         1     1    1    1

Amount of sport sentences: 4
Amount of not sport senteces: 1
Total amount of sentences: 5
    \end{Verbatim}

    \begin{tcolorbox}[breakable, size=fbox, boxrule=1pt, pad at break*=1mm,colback=cellbackground, colframe=cellborder]
\prompt{In}{incolor}{7}{\boxspacing}
\begin{Verbatim}[commandchars=\\\{\}]
\PY{c+c1}{\PYZsh{} creates a dictionary for each class}
\PY{k}{def} \PY{n+nf}{make\PYZus{}list}\PY{p}{(}\PY{n}{my\PYZus{}vector}\PY{p}{,} \PY{n}{my\PYZus{}x}\PY{p}{)}\PY{p}{:}
    \PY{n}{my\PYZus{}word\PYZus{}list} \PY{o}{=} \PY{n}{my\PYZus{}vector}\PY{o}{.}\PY{n}{get\PYZus{}feature\PYZus{}names}\PY{p}{(}\PY{p}{)}
    \PY{n}{my\PYZus{}count\PYZus{}list} \PY{o}{=} \PY{n}{my\PYZus{}x}\PY{o}{.}\PY{n}{toarray}\PY{p}{(}\PY{p}{)}\PY{o}{.}\PY{n}{sum}\PY{p}{(}\PY{n}{axis}\PY{o}{=}\PY{l+m+mi}{0}\PY{p}{)}
    \PY{n}{my\PYZus{}freq} \PY{o}{=} \PY{n+nb}{dict}\PY{p}{(}\PY{n+nb}{zip}\PY{p}{(}\PY{n}{my\PYZus{}word\PYZus{}list}\PY{p}{,} \PY{n}{my\PYZus{}count\PYZus{}list}\PY{p}{)}\PY{p}{)}
    \PY{k}{return} \PY{n}{my\PYZus{}word\PYZus{}list}\PY{p}{,} \PY{n}{my\PYZus{}count\PYZus{}list}\PY{p}{,} \PY{n}{my\PYZus{}freq}
\end{Verbatim}
\end{tcolorbox}

    \begin{tcolorbox}[breakable, size=fbox, boxrule=1pt, pad at break*=1mm,colback=cellbackground, colframe=cellborder]
\prompt{In}{incolor}{8}{\boxspacing}
\begin{Verbatim}[commandchars=\\\{\}]
\PY{c+c1}{\PYZsh{} create lists}

\PY{c+c1}{\PYZsh{} word\PYZus{}list\PYZus{}sport = word list [\PYZsq{}a\PYZsq{}, \PYZsq{}but\PYZsq{}, \PYZsq{}clean\PYZsq{}, \PYZsq{}forgettable\PYZsq{}, \PYZsq{}game\PYZsq{}, \PYZsq{}great\PYZsq{}, \PYZsq{}match\PYZsq{}, \PYZsq{}very\PYZsq{}]}
\PY{c+c1}{\PYZsh{} count\PYZus{}list\PYZus{}sport = occurence of words [2 1 2 1 2 1 1 1]}
\PY{c+c1}{\PYZsh{} freq\PYZus{}sport = combining the two to create a dictionary}
\PY{n}{word\PYZus{}list\PYZus{}sport}\PY{p}{,} \PY{n}{count\PYZus{}list\PYZus{}sport}\PY{p}{,} \PY{n}{freq\PYZus{}sport} \PY{o}{=} \PY{n}{make\PYZus{}list}\PY{p}{(}\PY{n}{vector\PYZus{}sport}\PY{p}{,} \PY{n}{X\PYZus{}sport}\PY{p}{)}
\PY{n}{word\PYZus{}list\PYZus{}not\PYZus{}sport}\PY{p}{,} \PY{n}{count\PYZus{}list\PYZus{}not\PYZus{}sport}\PY{p}{,} \PY{n}{freq\PYZus{}not\PYZus{}sport} \PY{o}{=} \PY{n}{make\PYZus{}list}\PY{p}{(}\PY{n}{vector\PYZus{}not\PYZus{}sport}\PY{p}{,} \PY{n}{X\PYZus{}not\PYZus{}sport}\PY{p}{)}

\PY{n+nb}{print}\PY{p}{(}\PY{l+s+sa}{f}\PY{l+s+s1}{\PYZsq{}}\PY{l+s+s1}{sport dictionary: }\PY{l+s+se}{\PYZbs{}n}\PY{l+s+si}{\PYZob{}freq\PYZus{}sport\PYZcb{}}\PY{l+s+se}{\PYZbs{}n}\PY{l+s+s1}{\PYZsq{}}\PY{p}{)}
\PY{n+nb}{print}\PY{p}{(}\PY{l+s+sa}{f}\PY{l+s+s1}{\PYZsq{}}\PY{l+s+s1}{not sport dictionary: }\PY{l+s+se}{\PYZbs{}n}\PY{l+s+si}{\PYZob{}freq\PYZus{}not\PYZus{}sport\PYZcb{}}\PY{l+s+se}{\PYZbs{}n}\PY{l+s+s1}{\PYZsq{}}\PY{p}{)}
\end{Verbatim}
\end{tcolorbox}

    \begin{Verbatim}[commandchars=\\\{\}]
sport dictionary:
\{'a': 3, 'but': 1, 'clean': 2, 'close': 1, 'forgettable': 1, 'game': 3, 'great':
1, 'match': 1, 'very': 2\}

not sport dictionary:
\{'election': 1, 'over': 1, 'the': 1, 'was': 1\}

    \end{Verbatim}

    \begin{tcolorbox}[breakable, size=fbox, boxrule=1pt, pad at break*=1mm,colback=cellbackground, colframe=cellborder]
\prompt{In}{incolor}{9}{\boxspacing}
\begin{Verbatim}[commandchars=\\\{\}]
\PY{c+c1}{\PYZsh{} calculate the probabilty of a word in a sentence of a class}
\PY{k}{def} \PY{n+nf}{calculate\PYZus{}prob}\PY{p}{(}\PY{n}{my\PYZus{}word\PYZus{}list}\PY{p}{,} \PY{n}{my\PYZus{}count\PYZus{}list}\PY{p}{)}\PY{p}{:}
    \PY{n}{my\PYZus{}prob} \PY{o}{=} \PY{p}{[}\PY{p}{]}
    \PY{k}{for} \PY{n}{my\PYZus{}word}\PY{p}{,} \PY{n}{my\PYZus{}count} \PY{o+ow}{in} \PY{n+nb}{zip}\PY{p}{(}\PY{n}{my\PYZus{}word\PYZus{}list}\PY{p}{,} \PY{n}{my\PYZus{}count\PYZus{}list}\PY{p}{)}\PY{p}{:}
        \PY{n}{my\PYZus{}prob}\PY{o}{.}\PY{n}{append}\PY{p}{(}\PY{n}{my\PYZus{}count} \PY{o}{/} \PY{n+nb}{len}\PY{p}{(}\PY{n}{my\PYZus{}word\PYZus{}list}\PY{p}{)}\PY{p}{)}
    \PY{n}{prob\PYZus{}dict} \PY{o}{=} \PY{n+nb}{dict}\PY{p}{(}\PY{n+nb}{zip}\PY{p}{(}\PY{n}{my\PYZus{}word\PYZus{}list}\PY{p}{,} \PY{n}{my\PYZus{}prob}\PY{p}{)}\PY{p}{)}
    \PY{k}{return} \PY{n}{prob\PYZus{}dict}
\end{Verbatim}
\end{tcolorbox}

    \begin{tcolorbox}[breakable, size=fbox, boxrule=1pt, pad at break*=1mm,colback=cellbackground, colframe=cellborder]
\prompt{In}{incolor}{10}{\boxspacing}
\begin{Verbatim}[commandchars=\\\{\}]
\PY{c+c1}{\PYZsh{} probabilities of the words in a class}
\PY{n}{prob\PYZus{}sport\PYZus{}dict} \PY{o}{=} \PY{n}{calculate\PYZus{}prob}\PY{p}{(}\PY{n}{word\PYZus{}list\PYZus{}sport}\PY{p}{,} \PY{n}{count\PYZus{}list\PYZus{}sport}\PY{p}{)}
\PY{n}{prob\PYZus{}not\PYZus{}sport\PYZus{}dict} \PY{o}{=} \PY{n}{calculate\PYZus{}prob}\PY{p}{(}\PY{n}{word\PYZus{}list\PYZus{}not\PYZus{}sport}\PY{p}{,} \PY{n}{count\PYZus{}list\PYZus{}not\PYZus{}sport}\PY{p}{)}
\PY{n+nb}{print}\PY{p}{(}\PY{l+s+sa}{f}\PY{l+s+s1}{\PYZsq{}}\PY{l+s+s1}{probabilites of words in sport sentences: }\PY{l+s+se}{\PYZbs{}n}\PY{l+s+si}{\PYZob{}prob\PYZus{}sport\PYZus{}dict\PYZcb{}}\PY{l+s+se}{\PYZbs{}n}\PY{l+s+s1}{\PYZsq{}}\PY{p}{)}
\PY{n+nb}{print}\PY{p}{(}\PY{l+s+sa}{f}\PY{l+s+s1}{\PYZsq{}}\PY{l+s+s1}{probabilites of words in not sport sentences: }\PY{l+s+se}{\PYZbs{}n}\PY{l+s+si}{\PYZob{}prob\PYZus{}not\PYZus{}sport\PYZus{}dict\PYZcb{}}\PY{l+s+s1}{\PYZsq{}}\PY{p}{)}
\end{Verbatim}
\end{tcolorbox}

    \begin{Verbatim}[commandchars=\\\{\}]
probabilites of words in sport sentences:
\{'a': 0.3333333333333333, 'but': 0.1111111111111111, 'clean':
0.2222222222222222, 'close': 0.1111111111111111, 'forgettable':
0.1111111111111111, 'game': 0.3333333333333333, 'great': 0.1111111111111111,
'match': 0.1111111111111111, 'very': 0.2222222222222222\}

probabilites of words in not sport sentences:
\{'election': 0.25, 'over': 0.25, 'the': 0.25, 'was': 0.25\}
    \end{Verbatim}

    \begin{tcolorbox}[breakable, size=fbox, boxrule=1pt, pad at break*=1mm,colback=cellbackground, colframe=cellborder]
\prompt{In}{incolor}{11}{\boxspacing}
\begin{Verbatim}[commandchars=\\\{\}]
\PY{c+c1}{\PYZsh{} all sentences again}
\PY{n}{docs} \PY{o}{=} \PY{p}{[}\PY{n}{row}\PY{p}{[}\PY{l+s+s1}{\PYZsq{}}\PY{l+s+s1}{sentence}\PY{l+s+s1}{\PYZsq{}}\PY{p}{]} \PY{k}{for} \PY{n}{index}\PY{p}{,} \PY{n}{row} \PY{o+ow}{in} \PY{n}{training\PYZus{}data}\PY{o}{.}\PY{n}{iterrows}\PY{p}{(}\PY{p}{)}\PY{p}{]}

\PY{c+c1}{\PYZsh{} vectorizer}
\PY{n}{vector} \PY{o}{=} \PY{n}{CountVectorizer}\PY{p}{(}\PY{n}{token\PYZus{}pattern}\PY{o}{=}\PY{l+s+sa}{r}\PY{l+s+s2}{\PYZdq{}}\PY{l+s+s2}{(?u)}\PY{l+s+s2}{\PYZbs{}}\PY{l+s+s2}{b}\PY{l+s+s2}{\PYZbs{}}\PY{l+s+s2}{w+}\PY{l+s+s2}{\PYZbs{}}\PY{l+s+s2}{b}\PY{l+s+s2}{\PYZdq{}}\PY{p}{)}

\PY{c+c1}{\PYZsh{} transform the sentences}
\PY{n}{X} \PY{o}{=} \PY{n}{vector}\PY{o}{.}\PY{n}{fit\PYZus{}transform}\PY{p}{(}\PY{n}{docs}\PY{p}{)}

\PY{c+c1}{\PYZsh{} counting the words}
\PY{n}{total\PYZus{}features} \PY{o}{=} \PY{n+nb}{len}\PY{p}{(}\PY{n}{vector}\PY{o}{.}\PY{n}{get\PYZus{}feature\PYZus{}names}\PY{p}{(}\PY{p}{)}\PY{p}{)}
\PY{n}{total\PYZus{}counts\PYZus{}features\PYZus{}sport} \PY{o}{=} \PY{n}{count\PYZus{}list\PYZus{}sport}\PY{o}{.}\PY{n}{sum}\PY{p}{(}\PY{n}{axis}\PY{o}{=}\PY{l+m+mi}{0}\PY{p}{)}
\PY{n}{total\PYZus{}counts\PYZus{}features\PYZus{}not\PYZus{}sport} \PY{o}{=} \PY{n}{count\PYZus{}list\PYZus{}not\PYZus{}sport}\PY{o}{.}\PY{n}{sum}\PY{p}{(}\PY{n}{axis}\PY{o}{=}\PY{l+m+mi}{0}\PY{p}{)}
                     
\PY{n+nb}{print}\PY{p}{(}\PY{l+s+sa}{f}\PY{l+s+s1}{\PYZsq{}}\PY{l+s+s1}{Amount of distinct words: }\PY{l+s+si}{\PYZob{}total\PYZus{}features\PYZcb{}}\PY{l+s+s1}{\PYZsq{}}\PY{p}{)}
\PY{n+nb}{print}\PY{p}{(}\PY{l+s+sa}{f}\PY{l+s+s1}{\PYZsq{}}\PY{l+s+s1}{Amount of distinct words in sport sentences: }\PY{l+s+si}{\PYZob{}total\PYZus{}counts\PYZus{}features\PYZus{}sport\PYZcb{}}\PY{l+s+s1}{\PYZsq{}}\PY{p}{)}
\PY{n+nb}{print}\PY{p}{(}\PY{l+s+sa}{f}\PY{l+s+s1}{\PYZsq{}}\PY{l+s+s1}{Amount of distinct words in not sport sentences: }\PY{l+s+si}{\PYZob{}total\PYZus{}counts\PYZus{}features\PYZus{}not\PYZus{}sport\PYZcb{}}\PY{l+s+s1}{\PYZsq{}}\PY{p}{)}
\end{Verbatim}
\end{tcolorbox}

    \begin{Verbatim}[commandchars=\\\{\}]
Amount of distinct words: 13
Amount of distinct words in sport sentences: 15
Amount of distinct words in not sport sentences: 4
    \end{Verbatim}

    \begin{tcolorbox}[breakable, size=fbox, boxrule=1pt, pad at break*=1mm,colback=cellbackground, colframe=cellborder]
\prompt{In}{incolor}{12}{\boxspacing}
\begin{Verbatim}[commandchars=\\\{\}]
\PY{c+c1}{\PYZsh{} a new sentence }
\PY{n}{new\PYZus{}sentence} \PY{o}{=} \PY{l+s+s1}{\PYZsq{}}\PY{l+s+s1}{Hermann plays a TT match}\PY{l+s+s1}{\PYZsq{}}

\PY{c+c1}{\PYZsh{} gets tokenized}
\PY{n}{new\PYZus{}word\PYZus{}list} \PY{o}{=} \PY{n}{word\PYZus{}tokenize}\PY{p}{(}\PY{n}{new\PYZus{}sentence}\PY{p}{)}
\end{Verbatim}
\end{tcolorbox}

    \begin{tcolorbox}[breakable, size=fbox, boxrule=1pt, pad at break*=1mm,colback=cellbackground, colframe=cellborder]
\prompt{In}{incolor}{13}{\boxspacing}
\begin{Verbatim}[commandchars=\\\{\}]
\PY{c+c1}{\PYZsh{} We\PYZsq{}re using laplace smoothing}
\PY{c+c1}{\PYZsh{} if a new word occurs the probability would be 0}
\PY{c+c1}{\PYZsh{} So every word counter gets incremented by one}
\PY{k}{def} \PY{n+nf}{laplace}\PY{p}{(}\PY{n}{freq}\PY{p}{,} \PY{n}{total\PYZus{}count}\PY{p}{,} \PY{n}{total\PYZus{}feat}\PY{p}{)}\PY{p}{:}
    \PY{n}{prob\PYZus{}sport\PYZus{}or\PYZus{}not} \PY{o}{=} \PY{p}{[}\PY{p}{]}
    \PY{k}{for} \PY{n}{my\PYZus{}word} \PY{o+ow}{in} \PY{n}{new\PYZus{}word\PYZus{}list}\PY{p}{:}
        \PY{k}{if} \PY{n}{my\PYZus{}word} \PY{o+ow}{in} \PY{n}{freq}\PY{o}{.}\PY{n}{keys}\PY{p}{(}\PY{p}{)}\PY{p}{:}
            \PY{n}{counter} \PY{o}{=} \PY{n}{freq}\PY{p}{[}\PY{n}{my\PYZus{}word}\PY{p}{]}
        \PY{k}{else}\PY{p}{:}
            \PY{n}{counter} \PY{o}{=} \PY{l+m+mi}{0}
        \PY{c+c1}{\PYZsh{} total\PYZus{}count is the amount of words in sport sentences and total\PYZus{}feat the total amount of words}
        \PY{n}{prob\PYZus{}sport\PYZus{}or\PYZus{}not}\PY{o}{.}\PY{n}{append}\PY{p}{(}\PY{p}{(}\PY{n}{counter} \PY{o}{+} \PY{l+m+mi}{1}\PY{p}{)} \PY{o}{/} \PY{p}{(}\PY{n}{total\PYZus{}count} \PY{o}{+} \PY{n}{total\PYZus{}feat}\PY{p}{)}\PY{p}{)}
    \PY{k}{return} \PY{n}{prob\PYZus{}sport\PYZus{}or\PYZus{}not}
\end{Verbatim}
\end{tcolorbox}

    \begin{tcolorbox}[breakable, size=fbox, boxrule=1pt, pad at break*=1mm,colback=cellbackground, colframe=cellborder]
\prompt{In}{incolor}{14}{\boxspacing}
\begin{Verbatim}[commandchars=\\\{\}]
\PY{c+c1}{\PYZsh{} probability for the new words}
\PY{n}{prob\PYZus{}new\PYZus{}sport} \PY{o}{=} \PY{n}{laplace}\PY{p}{(}\PY{n}{freq\PYZus{}sport}\PY{p}{,} \PY{n}{total\PYZus{}counts\PYZus{}features\PYZus{}sport}\PY{p}{,} \PY{n}{total\PYZus{}features}\PY{p}{)}
\PY{n}{prob\PYZus{}new\PYZus{}not\PYZus{}sport} \PY{o}{=} \PY{n}{laplace}\PY{p}{(}\PY{n}{freq\PYZus{}not\PYZus{}sport}\PY{p}{,} \PY{n}{total\PYZus{}counts\PYZus{}features\PYZus{}not\PYZus{}sport}\PY{p}{,} \PY{n}{total\PYZus{}features}\PY{p}{)}

\PY{n+nb}{print}\PY{p}{(}\PY{l+s+sa}{f}\PY{l+s+s1}{\PYZsq{}}\PY{l+s+s1}{probability that the word is in a sport sentece: }\PY{l+s+si}{\PYZob{}prob\PYZus{}new\PYZus{}sport\PYZcb{}}\PY{l+s+s1}{\PYZsq{}}\PY{p}{)}
\PY{n+nb}{print}\PY{p}{(}\PY{l+s+sa}{f}\PY{l+s+s1}{\PYZsq{}}\PY{l+s+s1}{probability that the word is in a not sport sentece: }\PY{l+s+si}{\PYZob{}prob\PYZus{}new\PYZus{}not\PYZus{}sport\PYZcb{}}\PY{l+s+s1}{\PYZsq{}}\PY{p}{)}
\end{Verbatim}
\end{tcolorbox}

    \begin{Verbatim}[commandchars=\\\{\}]
probability that the word is in a sport sentece: [0.03571428571428571,
0.03571428571428571, 0.14285714285714285, 0.03571428571428571,
0.07142857142857142]
probability that the word is in a not sport sentece: [0.058823529411764705,
0.058823529411764705, 0.058823529411764705, 0.058823529411764705,
0.058823529411764705]
    \end{Verbatim}

    \begin{tcolorbox}[breakable, size=fbox, boxrule=1pt, pad at break*=1mm,colback=cellbackground, colframe=cellborder]
\prompt{In}{incolor}{15}{\boxspacing}
\begin{Verbatim}[commandchars=\\\{\}]
\PY{c+c1}{\PYZsh{} multiplying the probabilities of each word}
\PY{n}{new\PYZus{}sport} \PY{o}{=} \PY{n+nb}{list}\PY{p}{(}\PY{n}{prob\PYZus{}new\PYZus{}sport}\PY{p}{)}
\PY{n}{sport\PYZus{}multiply\PYZus{}result} \PY{o}{=} \PY{l+m+mi}{1}
\PY{k}{for} \PY{n}{i} \PY{o+ow}{in} \PY{n+nb}{range}\PY{p}{(}\PY{l+m+mi}{0}\PY{p}{,} \PY{n+nb}{len}\PY{p}{(}\PY{n}{new\PYZus{}sport}\PY{p}{)}\PY{p}{)}\PY{p}{:}
    \PY{n}{sport\PYZus{}multiply\PYZus{}result} \PY{o}{*}\PY{o}{=} \PY{n}{new\PYZus{}sport}\PY{p}{[}\PY{n}{i}\PY{p}{]}

\PY{c+c1}{\PYZsh{} multiplying the result with the ratio of sports senteces to the total amount of sentences (here its 4/6)}
\PY{n}{sport\PYZus{}multiply\PYZus{}result} \PY{o}{*}\PY{o}{=} \PY{p}{(} \PY{n+nb}{len}\PY{p}{(}\PY{n}{tdm\PYZus{}sport}\PY{p}{)} \PY{o}{/} \PY{n+nb}{len}\PY{p}{(}\PY{n}{rows}\PY{p}{)} \PY{p}{)}

\PY{c+c1}{\PYZsh{} multiplying the probabilities of each word   }
\PY{n}{new\PYZus{}not\PYZus{}sport} \PY{o}{=} \PY{n+nb}{list}\PY{p}{(}\PY{n}{prob\PYZus{}new\PYZus{}not\PYZus{}sport}\PY{p}{)}
\PY{n}{not\PYZus{}sport\PYZus{}multiply\PYZus{}result} \PY{o}{=} \PY{l+m+mi}{1}
\PY{k}{for} \PY{n}{i} \PY{o+ow}{in} \PY{n+nb}{range}\PY{p}{(}\PY{l+m+mi}{0}\PY{p}{,} \PY{n+nb}{len}\PY{p}{(}\PY{n}{new\PYZus{}not\PYZus{}sport}\PY{p}{)}\PY{p}{)}\PY{p}{:}
    \PY{n}{not\PYZus{}sport\PYZus{}multiply\PYZus{}result} \PY{o}{*}\PY{o}{=} \PY{n}{new\PYZus{}not\PYZus{}sport}\PY{p}{[}\PY{n}{i}\PY{p}{]}
    
\PY{c+c1}{\PYZsh{} multiplying the result with the ratio of sports senteces to the total amount of sentences (here its 2/6)}
\PY{n}{not\PYZus{}sport\PYZus{}multiply\PYZus{}result} \PY{o}{*}\PY{o}{=} \PY{p}{(} \PY{n+nb}{len}\PY{p}{(}\PY{n}{tdm\PYZus{}not\PYZus{}sport}\PY{p}{)} \PY{o}{/} \PY{n+nb}{len}\PY{p}{(}\PY{n}{rows}\PY{p}{)} \PY{p}{)}
    
    
\end{Verbatim}
\end{tcolorbox}

    \begin{tcolorbox}[breakable, size=fbox, boxrule=1pt, pad at break*=1mm,colback=cellbackground, colframe=cellborder]
\prompt{In}{incolor}{16}{\boxspacing}
\begin{Verbatim}[commandchars=\\\{\}]
\PY{c+c1}{\PYZsh{} comparing whats more likely }

\PY{n+nb}{print}\PY{p}{(}\PY{l+s+sa}{f}\PY{l+s+s1}{\PYZsq{}}\PY{l+s+s1}{The probability of the sentence }\PY{l+s+s1}{\PYZdq{}}\PY{l+s+si}{\PYZob{}new\PYZus{}sentence\PYZcb{}}\PY{l+s+s1}{\PYZdq{}}\PY{l+s+s1}{:}\PY{l+s+se}{\PYZbs{}n}\PY{l+s+s1}{Sport vs not sport}\PY{l+s+se}{\PYZbs{}n}\PY{l+s+si}{\PYZob{}sport\PYZus{}multiply\PYZus{}result\PYZcb{}}\PY{l+s+s1}{ vs }\PY{l+s+si}{\PYZob{}not\PYZus{}sport\PYZus{}multiply\PYZus{}result\PYZcb{}}\PY{l+s+se}{\PYZbs{}n}\PY{l+s+se}{\PYZbs{}n}\PY{l+s+s1}{\PYZsq{}}\PY{p}{)}

\PY{k}{if} \PY{n}{not\PYZus{}sport\PYZus{}multiply\PYZus{}result} \PY{o}{\PYZlt{}} \PY{n}{sport\PYZus{}multiply\PYZus{}result}\PY{p}{:}
    \PY{n+nb}{print}\PY{p}{(}\PY{l+s+s1}{\PYZsq{}}\PY{l+s+s1}{Verdict: It}\PY{l+s+se}{\PYZbs{}\PYZsq{}}\PY{l+s+s1}{s probably a sports sentence!}\PY{l+s+s1}{\PYZsq{}}\PY{p}{)}
\PY{k}{else}\PY{p}{:}
    \PY{n+nb}{print}\PY{p}{(}\PY{l+s+s1}{\PYZsq{}}\PY{l+s+s1}{Verdict: It}\PY{l+s+se}{\PYZbs{}\PYZsq{}}\PY{l+s+s1}{s probably not a sport sentence!}\PY{l+s+s1}{\PYZsq{}}\PY{p}{)}
\end{Verbatim}
\end{tcolorbox}

    \begin{Verbatim}[commandchars=\\\{\}]
The probability of the sentence "Hermann plays a TT match":
Sport vs not sport
3.718688641637412e-07 vs 1.4085925554474852e-07


Verdict: It's probably a sports sentence!
    \end{Verbatim}

    \begin{tcolorbox}[breakable, size=fbox, boxrule=1pt, pad at break*=1mm,colback=cellbackground, colframe=cellborder]
\prompt{In}{incolor}{17}{\boxspacing}
\begin{Verbatim}[commandchars=\\\{\}]
\PY{c+c1}{\PYZsh{} print current date and time}

\PY{n+nb}{print}\PY{p}{(}\PY{l+s+s2}{\PYZdq{}}\PY{l+s+s2}{Date \PYZam{} Time:}\PY{l+s+s2}{\PYZdq{}}\PY{p}{,}\PY{n}{time}\PY{o}{.}\PY{n}{strftime}\PY{p}{(}\PY{l+s+s2}{\PYZdq{}}\PY{l+s+si}{\PYZpc{}d}\PY{l+s+s2}{.}\PY{l+s+s2}{\PYZpc{}}\PY{l+s+s2}{m.}\PY{l+s+s2}{\PYZpc{}}\PY{l+s+s2}{Y  }\PY{l+s+s2}{\PYZpc{}}\PY{l+s+s2}{H:}\PY{l+s+s2}{\PYZpc{}}\PY{l+s+s2}{M:}\PY{l+s+s2}{\PYZpc{}}\PY{l+s+s2}{S}\PY{l+s+s2}{\PYZdq{}}\PY{p}{)}\PY{p}{)}
\PY{c+c1}{\PYZsh{} end of import test}
\PY{n+nb}{print} \PY{p}{(}\PY{l+s+s2}{\PYZdq{}}\PY{l+s+s2}{*** End of Homework\PYZhy{}H3.2\PYZus{}Bayes\PYZhy{}Learning... ***}\PY{l+s+s2}{\PYZdq{}}\PY{p}{)}
\end{Verbatim}
\end{tcolorbox}

    \begin{Verbatim}[commandchars=\\\{\}]
Date \& Time: 24.08.2023  21:01:47
*** End of Homework-H3.2\_Bayes-Learning{\ldots} ***
    \end{Verbatim}


    % Add a bibliography block to the postdoc
    
    
    
\end{document}
