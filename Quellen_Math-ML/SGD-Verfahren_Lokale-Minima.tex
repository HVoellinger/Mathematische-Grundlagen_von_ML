\documentclass{article}
\usepackage{amsmath}
\usepackage{amsfonts}
\usepackage{amssymb}

\title{Minimierung einer Funktion mittels Stochastic Gradient Descent (SGD)}
\author{Hermann Völlinger}
\date{15.09.2024}

\begin{document}

\maketitle

\section*{Lösen des Beispiels mit Stochastic Gradient Descent (SGD)}

Wir wollen die Funktion
\[
f(x) = x^4 - 3x^2 + 2
\]
mit Hilfe des \textit{Gradient Descent (SGD)} - Verfahrens minimieren.\\[0.6cm]
Dieses vollständige LaTeX-Dokument beschreibt die Minimierung der Funktion 
 mithilfe des SGD-Verfahrens und zeigt die ersten paar Iterationen in einer Tabelle.\\
 
\subsection*{Schritt 1: Ableitung der Funktion}

Die Ableitung von \( f(x) \) ist:
\[
f'(x) = 4x^3 - 6x
\]

\subsection*{Schritt 2: SGD-Update-Regel}

Die Update-Regel des Gradientenabstiegs lautet:
\[
x_{\text{neu}} = x_{\text{alt}} - \eta f'(x_{\text{alt}})
\]
wobei \( \eta = 0.01 \) die Lernrate ist.

\subsection*{Schritt 3: Iterationen}

Wir starten mit einem zufälligen Wert \( x_0 = 0.5 \) und führen das SGD-Verfahren iterativ durch:

\begin{center}
\begin{tabular}{|c|c|c|c|c|}
\hline
Iteration \( n \) & \( x_n \)     & \( f'(x_n) \)         & \( x_{\text{neu}} \)         & \( f(x_{\text{neu}}) \) \\
\hline
1                 & 0.5           & \( f'(0.5) = -1.25 \) & \( 0.5 - 0.01 \times (-1.25) = 0.5125 \) & \( 1.761 \)             \\
2                 & 0.5125        & \( f'(0.5125) = -1.207 \) & \( 0.5125 - 0.01 \times (-1.207) = 0.52457 \) & \( 1.726 \)            \\
3                 & 0.52457       & \( f'(0.52457) = -1.166 \) & \( 0.52457 - 0.01 \times (-1.166) = 0.53624 \) & \( 1.692 \)            \\
4                 & 0.53624       & \( f'(0.53624) = -1.126 \) & \( 0.53624 - 0.01 \times (-1.126) = 0.5475 \)  & \( 1.659 \)            \\
\hline
\end{tabular}
\end{center}

\subsection*{Schritt 4: Stopkriterium}

Das Verfahren wird fortgesetzt, bis die Differenz zwischen \( f(x) \) in aufeinanderfolgenden Iterationen kleiner als ein festgelegter Schwellenwert \begin{Large}$ \epsilon $\end{Large} = 0.0001  ist.

\end{document}